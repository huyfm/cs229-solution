\item \points{40} {\bf Linear Classifiers (logistic regression and GDA)}

In this problem, we cover two probabilistic linear classifiers we have
covered in class so far. First, a discriminative linear classifier: logistic
regression. Second, a generative linear classifier: Gaussian discriminant
analysis (GDA). Both the algorithms find a linear decision boundary that
separates the data into two classes, but make different assumptions. Our goal
in this problem is to get a deeper understanding of the similarities and
differences (and, strengths and weaknesses) of these two algorithms.

For this problem, we will consider two datasets, along with starter codes provided in the following
files:
\begin{center}
\begin{itemize} %[label=\roman*.]
	\item \url{src/linearclass/ds1_{train,valid}.csv}
	\item \url{src/linearclass/ds2_{train,valid}.csv}
        \item \url{src/linearclass/logreg.py}
        \item \url{src/linearclass/gda.py}
\end{itemize}
\end{center}
Each file contains $\nexp$ examples, one example $(x^{(i)}, y^{(i)})$ per row.
In particular, the $i$-th row contains columns $x^{(i)}_1\in\Re$,
$x^{(i)}_2\in\Re$, and $y^{(i)}\in\{0, 1\}$. In the subproblems that follow, we
will investigate using logistic regression and Gaussian discriminant analysis
(GDA) to perform binary classification on these two datasets.

\begin{enumerate}
	\item \subquestionpoints{10}

%\textbf{TBD RETIRE THIS QUESTION? Q4c is a more generic version of this}

In lecture we saw the average empirical loss for logistic regression:
\begin{equation*}
	J(\theta)
	= -\frac{1}{\nexp} \sum_{i=1}^\nexp \left(y^{(i)}\log(h_{\theta}(x^{(i)}))
		+  (1 - y^{(i)})\log(1 - h_{\theta}(x^{(i)}))\right),
\end{equation*}
where $y^{(i)} \in \{0, 1\}$, $h_\theta(x) = g(\theta^T x)$ and
$g(z) = 1 / (1 + e^{-z})$.

Find the Hessian $H$ of this function, and show that for any vector $z$, it
holds true that
%
\begin{equation*}
    z^T H z \ge 0.
\end{equation*}
%
{\bf Hint:} You may want to start by showing that
$\sum_i\sum_j z_i x_i x_j z_j = (x^Tz)^2 \geq 0$. Recall also that
$g'(z) = g(z)(1-g(z))$.

{\bf Remark:} This is one of the standard ways of showing that the matrix $H$
is positive semi-definite, written ``$H \succeq 0$.''  This implies that $J$ is
convex, and has no local minima other than the global one. If you have some
other way of showing $H \succeq 0$, you're also welcome to use your method
instead of the one above.


        \ifnum\solutions=1 {
            \begin{answer}
Compute the gradient of the cost function $J(\theta)$, we have:
%
\begin{align}
	\nabla_{\theta}J(\theta) 
	&= \frac{1}{n} \sum \limits_{i = 1}^{n} \left(-y^{(i)} \nabla_{\theta}\log(g(\theta^{T}\xsi)) - (1 - y^{(i)}) \nabla_{\theta}\log(1 - g(\theta^{T}\xsi))\right) \\
	&= \frac{1}{n} \sum_{i = 1}^{n} \left(-y^{(i)}(1 - g(\theta^{T}\xsi))\xsi + (1 - y^{(i)}) g(\theta^{T}\xsi) x^{(i)}\right) \\
	&= \frac{1}{n} \sum \limits_{i = 1}^{n} (g(\theta^{T}\xsi) - y^{(i)}) x^{(i)}
\end{align}
%
Now, take the second derivative of the gradient, we obtain the hessian matrix:
%
\begin{align}
	H(J)(\theta) 
	= \frac{1}{n} \sum \limits_{i = 1}^{n} g(\theta^{T}\xsi)(1 - g(\theta^{T}\xsi)) x^{(i)} {x^{(i)}}^T
\end{align}
%
Remember that $h_{\theta}(x^{(i)}) = g(\theta^{T}\xsi)$.
For any vector $z \in \mathbb{R}^{d + 1}$, let's consider:
%
\begin{align}
	z^{T} H(J)(\theta) z
	&= \frac{1}{n} \sum \limits_{i = 1}^{n} h_{\theta}(x^{(i)})(1 - h_{\theta}(x^{(i)})) (z^{T} x^{(i)})^2
\end{align}
%
All the terms are non-negative as $h_{\theta}(x^{(i)})$ is a probability and squares are non-negative. This implies that $z^{T} H(J)(\theta) z \ge 0$ for all vector $z \in \mathbb{R}^{d + 1}$, hence $H(J)(\theta) \succeq 0$. \\
\end{answer}





















        } \fi

	\item \subquestionpoints{5} \textbf{Coding problem.}
Follow the instructions in \texttt{src/linearclass/logreg.py} to train a
logistic regression classifier using Newton's Method.
Starting with $\theta = \vec{0}$, run Newton's Method until the updates to
$\theta$ are small: Specifically,  train until the first iteration $k$ such
that $\|\theta_{k} - \theta_{k-1}\|_1 < \epsilon$, where
$\epsilon = 1\times 10^{-5}$. Make sure to write your model's predicted probabilities on
the validation set to the file specified in the code.

Include a plot of the \textbf{validation data} with $x_1$ on the horizontal axis and $x_2$ on the vertical axis.
To visualize the two classes, use a different symbol for examples $x^{(i)}$
with $y^{(i)} = 0$ than for those with $y^{(i)} = 1$. On the same figure, plot the decision boundary
found by logistic regression (i.e, line corresponding to $p(y|x) = 0.5$).


        \ifnum\solutions=1 {
            \begin{answer}
\begin{figure}[H]
	\centering
	\begin{subfigure}[H]{0.45\linewidth}
		\includegraphics[width=\linewidth]{logreg_1}
		\caption{Dataset 1.}
	\end{subfigure}
	\begin{subfigure}[H]{0.45\linewidth}
		\includegraphics[width=\linewidth]{logreg_2}
		\caption{Dataset 2.}
	\end{subfigure}
	\caption{Run logistic regression on two given datasets.}
\end{figure}
\end{answer}


        } \fi


	\input{linearclass/03-gda}
        \ifnum\solutions=1 {
            \begin{answer}	
We have:
%
\begin{align}
	p(y = 1|x; \theta, \mu_0, \mu_1, \Sigma)
	&= \frac{p(y =1)p(x|y = 1)}{p(y = 0)p(x|y = 0) + p(y=1)p(x|y = 1)} \\
	&= \frac{1}{1 + \dfrac{p(y = 0)p(x|y = 0)}{p(y=1)p(x|y = 1)}}
\end{align}
%
Let's consider the fraction in the denominator, we have:
%
\begin{align}
	& \dfrac{p(y = 0)p(x|y = 0)}{p(y=1)p(x|y = 1)} \\
	&= \frac{\dfrac{1 - \phi}{(2\pi)^{d/2}|\Sigma|^{1/2}} \exp((x - \mu_1)^T \Sigma^{-1} (x - \mu_1))}{\dfrac{\phi}{(2\pi)^{d/2}|\Sigma|^{1/2}} \exp((x - \mu_0)^T \Sigma^{-1} (x - \mu_0))} \\
	&= \left(\frac{1 - \phi}{\phi}\right) \exp\left( \frac{1}{2} (x - \mu_1)^T \Sigma^{-1} (x - \mu_1) - \frac{1}{2} (x - \mu_0)^T \Sigma^{-1} (x - \mu_0) \right)
\end{align}
% 
Write $x - \mu_1$ as $x - \mu_0 + \mu_0 - \mu_1$ and group similar terms together, the exponent in the above expression (line (12)) can be derived further as follows:
%
\begin{align}
& \frac{1}{2} (x - \mu_1)^T \Sigma^{-1} (x - \mu_0 + \mu_0 - \mu_1) - \frac{1}{2} (x - \mu_0)^T \Sigma^{-1} (x - \mu_0) \\
&= -\frac{1}{2} (\mu_1 - \mu_0)^T \Sigma^{-1} (x - \mu_0) - \frac{1}{2} (\mu_1 - \mu_0)^T \Sigma^{-1} (x - \mu_1) \\
&= -(\mu_1 - \mu_0)^T \Sigma^{-1} (x - \frac{\mu_0 + \mu_1}{2})
\end{align}
%
Putting everything together, we obtain the above fraction equal to:
$$ \exp\left\{-(\mu_1 - \mu_0)^T \Sigma^{-1}x - \left( \log \left( \frac{\phi}{1 - \phi} \right) - \frac{(\mu_1 - \mu_0)^T \Sigma^{-1} (\mu_0 + \mu_1)}{2} \right) \right\} $$
If we let $$\theta = (\mu_1 - \mu_0)^T \Sigma^{-1}x,$$ $$\theta_0 = \log \left( \dfrac{\phi}{1 - \phi} \right) - \dfrac{(\mu_1 - \mu_0)^T \Sigma^{-1} (\mu_0 + \mu_1)}{2},$$ then the posterior probability can be re-written as $\dfrac{1}{1 + \exp(-\theta^T - \theta_0)}$. \\[2pt]
If the threshold to make prediction is $0.5$, the decision boundary will have an equation $\theta^T x + \theta_0 = 0$. \\
\end{answer}

        }\fi

	\item \subquestionpoints{7} Given the dataset, we claim that the maximum
  likelihood estimates of the parameters are given by
  \begin{eqnarray*}
    \phi &=& \frac{1}{\nexp} \sum_{i=1}^\nexp 1\{y^{(i)} = 1\} \\
\mu_{0} &=& \frac{\sum_{i=1}^\nexp 1\{y^{(i)} = {0}\} x^{(i)}}{\sum_{i=1}^\nexp
1\{y^{(i)} = {0}\}} \\
\mu_1 &=& \frac{\sum_{i=1}^\nexp 1\{y^{(i)} = 1\} x^{(i)}}{\sum_{i=1}^\nexp 1\{y^{(i)}
= 1\}} \\
\Sigma &=& \frac{1}{\nexp} \sum_{i=1}^\nexp (x^{(i)} - \mu_{y^{(i)}}) (x^{(i)} -
\mu_{y^{(i)}})^T
  \end{eqnarray*}
  The log-likelihood of the data is
  \begin{eqnarray*}
\ell(\phi, \mu_{0}, \mu_1, \Sigma) &=& \log \prod_{i=1}^\nexp p(x^{(i)} , y^{(i)};
\phi, \mu_{0}, \mu_1, \Sigma) \\
&=& \log \prod_{i=1}^\nexp p(x^{(i)} | y^{(i)}; \mu_{0}, \mu_1, \Sigma) p(y^{(i)};
\phi).
  \end{eqnarray*}
By maximizing $\ell$ with respect to the four parameters,
prove that the maximum likelihood estimates of $\phi$, $\mu_{0}, \mu_1$, and
$\Sigma$ are indeed as given in the formulas above.  (You may assume that there
is at least one positive and one negative example, so that the denominators in
the definitions of $\mu_{0}$ and $\mu_1$ above are non-zero.)


        \ifnum\solutions=1 {
            \begin{answer}
The log-likelihood function of the data is:

\begin{align}
	\ell(\phi, \mu_0, \mu_1, \Sigma)
	&= \log \prod \limits_{i = 1}^{n} p(x^{(i)}|y^{(i)})p(y^{(i)}) \\
	&= \sum \limits_{i = 1}^{n} (\log p(x^{(i)}|y^{(i)}) \log p(y^{(i)})) \\
	&= 
	\begin{aligned}[t]
		& \sum \limits_{i = 1}^{n} \bigg( -\log (2\pi)^{d/2} - \frac{1}{2}\log |\Sigma| - \frac{1}{2}(x^{(i)} - \mu_{y^{(i)}})^T \Sigma^{-1} (x^{(i)} - \mu_{y^{(i)}}) \\
		& + y^{(i)}\log\phi + (1 - y^{(i)})\log(1 - \phi) \bigg)
	\end{aligned}
\end{align}

The maximum likelihood estimate of  $\theta = (\phi, \mu_0, \mu_1, \Sigma)$ are the value which maximizes the log likelihood function. We will solve the equation $\nabla \ell(\theta) = 0$ in order to obtain them. We have:

\begin{align}
	\nabla_{\phi} \ell(\theta) 
	&= \sum \limits_{i = 1}^{n} \left( \frac{y^{(i)}}{\theta} - \frac{1 - y^{(i)}}{1 - \phi} \right)
	= 0
\end{align}

Solve the above equation gives:
$$\hat{\phi} = \frac{1}{n} \sum \limits_{i = 1}^{n} y^{(i)} = \frac{1}{n} \sum \limits_{i = 1}^{n} 1\{y^{(i)} = 1\}$$

Let's consider now:

\begin{align}
	\nabla_{\mu_0} \ell(\theta) &
	= \nabla_{\mu_0} \left( \sum \limits_{i = 1}^{n} - \frac{1}{2} 1\{y^{(i)} = 0\} (x^{(i)} - \mu_0)^T \Sigma^{-1} (x^{(i)} - \mu_0) \right) \\
	&= - \Sigma^{-1} \sum \limits_{i = 1}^{n} 1\{y^{(i)} = 0\}(\mu_0 - x^{i})
\end{align}

Solve the equation $\nabla_{\mu_0} \ell(\theta) = 0$ gives: $$\widehat{\mu_0} = \frac{ \sum_{i = 1}^{n} 1\{y^{(i)} = 0\}x^{(i)}}{\sum_{i = 1}^{n} 1\{y^{(i)} = 0\}}$$
Similarly, we also obtain $$\widehat{\mu_1} = \frac{ \sum_{i = 1}^{n} 1\{y^{(i)} = 1\}x^{(i)}}{\sum_{i = 1}^{n} 1\{y^{(i)} = 1\}}$$ when solving the corresponding equation $\nabla_{\mu_1} \ell(\theta) = 0$.

Lastly, we compute:

\begin{align}
	\nabla_{\Sigma}\ell(\theta) 
	&= \nabla_{\Sigma} \sum \limits_{i = 1}^{n} \left(- \frac{1}{2}\log |\Sigma| - \frac{1}{2}(x^{(i)} - \mu_{y^{(i)}})^T \Sigma^{-1} (x^{(i)} - \mu_{y^{(i)}}) \right) \\
	&= \sum \limits_{i = 1}^{n} \left( -\frac{1}{2} \Sigma^{-1} + \frac{1}{2}  \Sigma^{-1} (x^{(i)} - \mu_{y^{(i)}}) (x^{(i)} - \mu_{y^{(i)}})^T \Sigma^{-1} \right) \\
	&= \frac{1}{2} \Sigma^{-1}\left( \sum \limits_{i = 1}^{n} (x^{(i)} - \mu_{y^{(i)}}) (x^{(i)} - \mu_{y^{(i)}})^T \Sigma^{-1} - n \right)
\end{align}

Solve the equation $\nabla_{\Sigma} \ell(\theta) = 0$, we get:
$$\widehat\Sigma = \frac{1}{n} \sum \limits_{i = 1}^{n} (x^{(i)} - \mu_{y^{(i)}}) (x^{(i)} - \mu_{y^{(i)}})^T$$
In the above derivation of $\nabla_{\Sigma} \ell(\theta)$, we used 2 identities:

\begin{align}
\nabla_{\Sigma}\log|\Sigma| &= (\Sigma^{-1})^T \\
\nabla_{\Sigma} z^T\Sigma^{-1} z &= -\Sigma^{-1} z z^T \Sigma^{-1}, 
\end{align}

where $z$ is not a function of $\Sigma$. To prove the second identity, we take the differential:
\begin{align}
	d(z^T \Sigma^{-1} z) 
	&= d(\text{Tr}[z^T \Sigma^{-1} z]) \\
	&= \text{Tr}[z z^T d\Sigma^{-1}] \\
	&= \text{Tr}[-z z^T \Sigma^{-1} (d\Sigma) \Sigma^{-1}] \\
	&= \text{Tr}[-\Sigma^{-1} z z^T \Sigma^{-1} d\Sigma]
\end{align}
Here, we replace $ d\Sigma^{-1} = -\Sigma^{-1}(d\Sigma)\Sigma^{-1} $ (derived from the identity $\Sigma \Sigma^{-1} = I)$. Thus, $\nabla_{\Sigma} z^T\Sigma^{-1} z = -\Sigma^{-1} z z^T \Sigma^{-1}$. \\
\end{answer}

        } \fi

	\item \subquestionpoints{5} \textbf{Coding problem.}
In \texttt{src/linearclass/gda.py}, fill in the code to
calculate $\phi$, $\mu_{0}$, $\mu_{1}$, and $\Sigma$, use these parameters
to derive $\theta$, and use the resulting GDA model to make predictions on the
validation set. Make sure to write your model's predictions on
the validation set to the file specified in the code.

Include a plot of the \textbf{validation data} with $x_1$ on the horizontal axis and $x_2$ on the vertical axis.
To visualize the two classes, use a different symbol for examples $x^{(i)}$
with $y^{(i)} = 0$ than for those with $y^{(i)} = 1$. On the same figure, plot the decision boundary
found by GDA (i.e, line corresponding to $p(y|x) = 0.5$).


        \ifnum\solutions=1 {
            \begin{answer}

\begin{figure}[H]
	\centering
	\begin{subfigure}[H]{0.45\linewidth}
		\includegraphics[width=\linewidth]{gda_1}
		\caption{Dataset 1.}
	\end{subfigure}
	\begin{subfigure}[H]{0.45\linewidth}
		\includegraphics[width=\linewidth]{gda_2}
		\caption{Dataset 2.}
	\end{subfigure}
	\caption{Run GDA on two given datasets.}
\end{figure}

\end{answer}

        } \fi

	\item \subquestionpoints{2}
For Dataset 1, compare the validation set plots obtained in part (b) and part (e)
from logistic regression and GDA respectively, and briefly comment on your observation
in a couple of lines.


        \ifnum\solutions=1 {
            \begin{answer}
\begin{figure}[H]
	\centering
	\begin{subfigure}[H]{0.45\linewidth}
		\includegraphics[width=\linewidth]{logreg_1}
		\caption{Logistic regression}
	\end{subfigure}
	\begin{subfigure}[H]{0.45\linewidth}
		\includegraphics[width=\linewidth]{gda_1}
		\caption{GDA}
	\end{subfigure}
	\caption{Results on the Dataset 1.}
\end{figure}
In this case, the logistic regression model obtained a better decision boundary than the other. Its separating line was quite reasonable. \\	
\end{answer}

        } \fi

	\item \subquestionpoints{5}
Repeat the steps in part (b) and part (e) for Dataset 2. Create similar plots on
the \textbf{validation set} of Dataset 2 and include those plots in your writeup.

On which dataset does GDA seem to
perform worse than logistic regression? Why might this be the case?


        \ifnum\solutions=1{
            %\documentclass{article}[10pt,a4paper]
%\usepackage{amsfonts, amsmath, amssymb}
%\usepackage[margin=1in]{geometry}
%\usepackage{graphicx, subcaption}

\begin{answer}
	
	\begin{figure}[H]
		\centering
		\begin{subfigure}[H]{0.45\linewidth}
			\includegraphics[width=\linewidth]{logreg_2}
			\caption{Logistic regression}
		\end{subfigure}
		\begin{subfigure}[H]{0.45\linewidth}
			\includegraphics[width=\linewidth]{gda_2}
			\caption{GDA}
		\end{subfigure}
		\caption{Results on the Dataset 2.}
	\end{figure}
	In the validation data set 2, both models gave rather similar results, though the logistic regression model was slightly better. Overall, both learned decision boundaries are good on this data set. \\
	As we have observed from the previous question, GDA performed worse than logistic regression on data set 1. Two classes in data set 2 don't look like Gaussian distributed.
	
\end{answer}


        }\fi

	\item \points{1} For the dataset where GDA performed worse in
parts (f) and (g), can you find a transformation of the $x^{(i)}$'s such
that GDA performs significantly better? What might this transformation be?


        \ifnum\solutions=1{
            \begin{answer}
We can apply a log transformation $\log(\cdot)$ to the $2^{nd}$ feature. Transformed data will be more likely to come from a Gaussian. \\
\end{answer}

        }\fi

\end{enumerate}
